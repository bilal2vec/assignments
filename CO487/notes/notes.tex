\documentclass[11pt]{article}
\usepackage[utf8]{inputenc}
\usepackage[english]{babel}
\usepackage{bilal2vec}
\usepackage{minted}

\title{CO 487 - Notes}
\author{Bilal Khan (b54khan)\\
\href{mailto:bilal.khan@student.uwaterloo.ca}{bilal.khan@student.uwaterloo.ca}}
\date{\today}

\begin{document}

\maketitle

\tableofcontents

\section{Symmetric Encryption}

\paragraph{Kerckhoff's principle} The adversary knows everything about the SKES, except the particular key $k$ chosen by Alice and Bob. (Avoid security by obscurity!!)

\paragraph{Adversary's Goals in Breaking Symmetric Key Encryption:}
\begin{enumerate}
    \item Recover the secret key.
    \item Systematically recover plaintext from ciphertext (without necessarily learning the secret key).
    \item Learn some partial information about the plaintext from the ciphertext (other than its length).
\end{enumerate}

\paragraph{Security Definitions}
\begin{itemize}
    \item If the adversary can achieve goals 1 or 2, the SKES is said to be \textit{totally insecure} (or \textit{totally broken}).
    \item If the adversary cannot learn any partial information about the plaintext from the ciphertext (except possibly its length), the SKES is said to be \textit{semantically secure}.
    \item Hiding length information is very hard. This topic falls under the heading of \textit{traffic analysis}.
\end{itemize}

\paragraph{Security by obscurity} The adversary knows everything about the SKES, except the particular key $k$ chosen by Alice and Bob.


\paragraph{Passive attacks}
\begin{itemize}
    \item Ciphertext-only attack: The adversary only sees some encrypted ciphertexts.
    \item Known-plaintext attack: The adversary also knows some plaintext and the corresponding ciphertext.
\end{itemize}

\paragraph{Active attacks:}
\begin{itemize}
    \item Chosen-plaintext attack: The adversary can also choose some plaintext(s) and obtain the corresponding ciphertext(s).
    \item Chosen-ciphertext attack: The adversary can also choose some ciphertext(s) and obtain the corresponding plaintext(s). Includes the powers of chosen-plaintext attack.
\end{itemize}

\paragraph{Limits on Computational Powers}

\begin{itemize}
    \item Information-theoretic security: The adversary has infinite computational resources.
    \item Complexity-theoretic security: The adversary is a "polynomial-time Turing machine".
    \item Computational security: The adversary has $X$ number of real computers/workstations/supercomputers. ("computationally bounded"). Equivalently, the adversary can do $X$ basic operations, e.g., CPU cycles.
\end{itemize}

\paragraph{Pseudorandom bit generator (PRBG)} is a deterministic algorithm that takes as input a short random seed, and outputs a longer pseudorandom sequence, also known as a keystream.

\paragraph{Indistinguishability} The output sequence should be indistinguishable from a random sequence.

\paragraph{Unpredictability} If an adversary knows a portion $c_1$ of ciphertext and the corresponding plaintext $m_1$, then she can easily find the corresponding portion $k_1 = c_1 \oplus m_1$ of the output sequence. Thus, given portions of the output sequence, it should be infeasible to learn any information about the rest of the output sequence.

\paragraph{Stream cipher period} For a good stream cipher we need that the period of the keystream output sequence is larger than the length of the plaintext.

\paragraph{Stream ciphers} A5/1, RC4, Salsa20, ChaCha20, main idea: use a PRBG to generate a keystream, and then XOR the keystream with the plaintext to get the ciphertext.

\paragraph{Security}
\begin{itemize}
    \item Diffusion: each ciphertext bit should depend on all plaintext and all key bits.
    \item Confusion: the relationship between key bits, plaintext bits, and ciphertext bits should be complicated.
    \item Cascade or avalanche effect: changing one bit of plaintext or key should change each bit of ciphertext with probability about 50\%
    \item Key length: should be small, but large enough to preclude exhaustive key search.
\end{itemize}


\paragraph{AES/DES} substitution-permutation network. round key xor-ed into the state. DES uses a Feistel network.

\paragraph{ECB mode} encrypt each block independently. violates semantic security. padding required.

\paragraph{CBC mode} $C_i = E(K, C_{i-1} \oplus P_i), C_0 = IV$. padding required.

\paragraph{CFB mode} $C_i = E(K, C_{i-1}) \oplus P_i, C_0 = IV$. semantically secure.

\paragraph{OFB mode} $O_i = E(K, O_{i-1}), C_i = O_i \oplus P_i, O_0 = IV$. xor the plaintext into the ciphetext "stream".

\section{Hash Functions} MD4, MD5, SHA-1, SHA-2, SHA-3

\paragraph{Preimage resistance} Hard to invert given just an output.

\paragraph{2nd preimage resistance} Hard to find a second input that has the same hash value as a given first input.

\paragraph{Collision resistance} Hard to find two different inputs that have the same hash values.

\section{MACs} provides confidentiality (semantic security) and integrity (EUF-CMA).

\paragraph{HMAC} $HMAC(K, m) = H((K \oplus opad) \parallel H((K \oplus ipad) \parallel m))$, secure against length extension attacks.

\paragraph{MAC-then-encrypt / MAC-and-encrypt}: Insecure, no guarantee MAC doesn't leak plaintext bits

\paragraph{Encrypt-then-MAC}: Secure, but requires padding.

\section{PRNGs} Indistinguishability.

$$PRG : \{0, 1\}^\lambda \to \{0, 1\}^\ell$$
$$PRF : \{0, 1\}^\lambda \times \{0, 1\}^* \to \{0, 1\}^\ell$$
$$KDF : \{0, 1\}^\lambda \times \{0, 1\}^* \to \{0, 1\}^\ell$$

\paragraph{Pseudorandom generator} deterministic function that takes as input a random seed $k \in \{0, 1\}^\lambda$ and outputs a random-looking binary string of length $\ell$.

Expanding a strong short key into a long key (e.g., stream cipher): $HMAC_k(1), HMAC_k(2), HMAC_k(3), \ldots$

\paragraph{Pseudorandom function} deterministic function that takes as input a random seed $k \in \{0, 1\}^\lambda$ and a (non-secret) label in $\{0, 1\}^*$ and outputs a random-looking binary string of length $\ell$.

Deriving many keys from a single key: $HMAC_k(label)$

\paragraph{Key derivation function} deterministic function that takes as input a random seed $k \in \{0, 1\}^\lambda$ and a (non-secret) label in $\{0, 1\}^*$ and outputs a random-looking binary string of length $\ell$.

Turn longer non-uniform keys into shorter uniform keys $HMAC_{label}(k)$

\section{Passwords}

\begin{itemize}
  \item Knowledge-Based (Something you know)
  \item Object-Based (Something you have)
  \item ID-Based (Something you are)
  \item Location-based (Somewhere you are)
\end{itemize}

\paragraph{Rainbow tables} Hash begin/end of chain instead of every hash.

\paragraph{Salting} protect against rainbow tables.

\section{Public Key Cryptography} allows for non-repudiation.

\paragraph{RSA}

$$E((n, e), m) = m^e \mod n$$
$$D((n, d), c) = c^d \mod n$$

\begin{enumerate}
    \item Choose random primes $p$ and $q$ with $\log_2 p \approx \log_2 q \approx \ell/2$.
    \item Compute $n = pq$ and $\varphi(n) = (p-1)(q-1)$.
    \item Choose an integer $e$ with $1 < e < \varphi(n)$ and $\gcd(e, \varphi(n)) = 1$.
    \item Compute $d = e^{-1}$ in $\mathbb{Z}_{\varphi(n)}$.
    \begin{enumerate}
        \item Use the Extended Euclidean Algorithm to compute $d$.\\
        If the Extended Euclidean Algorithm succeeds, then you are guaranteed that\\
        $\gcd(e, \varphi(n)) = 1$.
    \end{enumerate}
    \item[] Note: $de \equiv 1 \pmod{\varphi(n)}$ by definition of $e^{-1}$.
    \item The public key is $(n, e)$.
    \item The private key is $(n, d)$.
\end{enumerate}

\paragraph{Fermat's Little Theorem}

\begin{theorem}[Fermat's Little Theorem]
Let $p$ be a prime. For all integers $a$, it holds that
\[ a^p \equiv a \pmod{p} \]

Moreover, if $a$ is coprime to $p$, then
\[ a^{p-1} \equiv 1 \pmod{p} \]
\end{theorem}

\paragraph{Modular operations} addition/subtraction is $O(l)$, multiplication/inversion is $O(l^2)$, exponentiation is $O(l^3)$.

\paragraph{Square-and-multiply} $a^b \mod n$, $\log n$ iterations of modular multiplications that each take $O(l^2)$ time, so $O(l^3)$ time total.

\section{Diffie-Hellman}

$$\mathbb{Z}_n^* = \{1, 2, \ldots, n-1 : \gcd(a, n) = 1\}$$

\paragraph{Order of $x \in \mathbb{Z}_n^*$} Smallest positive integer $k$ such that $x^k \equiv 1 \pmod{n}$.

\paragraph{Generator of $\mathbb{Z}_n^*$} An element $g \in \mathbb{Z}_n^*$ such that every $y \in \mathbb{Z}_n^*$ can be written as $g^k$ for some $k \in \mathbb{Z}$.

\paragraph{$\mathbb{Z}_p^*$} for any prime $p$, every generator has order $p-1$.

\paragraph{Group} a set $G$ with a binary operation $\cdot$ that satisfies:
\begin{itemize}
  \item Associativity: for all $a, b, c \in G$, $(a \cdot b) \cdot c = a \cdot (b \cdot c)$
  \item Identity: there exists an element $\mathcal{O} \in G$ such that for all $a \in G$, $a \cdot \mathcal{O} = \mathcal{O} \cdot a = a$
  \item Inverses: for all $a \in G$, there exists an element $a^{-1} \in G$ such that $a \cdot a^{-1} = a^{-1} \cdot a = \mathcal{O}$
\end{itemize}

\paragraph{Cyclic group} if it can be generated by a single element

\paragraph{Diffie-Hellman key exchange} has forward secrecy.

\begin{itemize}
  \item Alice and Bob agree on a group $G$ and a generator $g$ of $G$
  \item Alice chooses a random integer $a \in \mathbb{Z}_n^*$ and computes $g^a \mod p$
  \item Bob chooses a random integer $b \in \mathbb{Z}_n^*$ and computes $g^b \mod p$
  \item Alice and Bob exchange $g^a \mod p$ and $g^b \mod p$
  \item Alice computes $(g^b)^a \mod p$
  \item Bob computes $(g^a)^b \mod p$
  \item Alice and Bob use $g^{ab} \mod p$ as their shared secret key
\end{itemize}

\paragraph{CDH} $g^{ab} \mod p$ is hard to compute given $g$, $g^a \mod p$, and $g^b \mod p$.

\paragraph{DDH} hard to distinguish between $g^{ab} \mod p$ and a random element of $\mathbb{Z}_p^*$.

\paragraph{DLOG} given $g$ and $g^a \mod p$, it is hard to compute $a$.

\paragraph{ElGamal}

\begin{itemize}
  \item Publish one half of the Diffie-Hellman key exchange as the public key.
  \item Include the other half of the key exchange as the first half of the ciphertext.
  \item Encrypt the plaintext under the shared secret key as the second half of the ciphertext
\end{itemize}

\begin{itemize}
  \item Alice and Bob agree on a large prime $p$ and a $g \in \mathbb{Z}_p^*$ of large prime order $q$
  \item Choose a random $x \in \mathbb{Z}_q^*$
  \item Set $k_{\text{pubkey}} = g^x \mod p$
  \item Set $k_{\text{privkey}} = x$
  \item To encrypt a message $m \in \mathbb{Z}_p^*$, choose a random $r \in \mathbb{Z}_q^*$
  \item $E(k_{\text{pubkey}}, m) = (g^r \mod p, m \cdot k_{\text{pubkey}}^r \mod p)$
  \item To decrypt a ciphertext $(c_1, c_2)$, compute $m = c_2 \cdot (c_1^x)^{-1} \mod p$
\end{itemize}

\paragraph{Attack types}

\paragraph{Passive attacks}
\begin{itemize}
  \item Key-only attack: The adversary is given the public key(s).
  We always assume the adversary has the public key(s).
  \item Chosen-plaintext attack: The adversary can choose some plaintext(s) and obtain the
  corresponding ciphertext(s). Equivalent to a key-only attack.
\end{itemize}

\paragraph{Active attacks}
\begin{itemize}
  \item Non-adaptive chosen-ciphertext attack: The adversary can choose some ciphertext(s)
    and obtain the corresponding plaintext(s).
  \item (Adaptive) chosen-ciphertext attack: Same as above, except the adversary can also
    iteratively choose which ciphertexts to decrypt, based on the results of previous queries
\end{itemize}

\paragraph{Totally insecure} if the adversary can obtain the private key.

\paragraph{One-way} if the adversary cannot decrypt a given ciphertext.

\paragraph{Semantically secure} if the adversary cannot learn any partial information about a message.

\paragraph{RSA security} Hard if integer factorization is hard. Not semantically secure since its deterministic. No deterministic encryption is semantically secure, but randomized encryption is not sufficient for semantic security.

\paragraph{Subexponential time} representated as a pair of $L_n[\alpha, c]$ where the value of $\alpha$ determines the hardness.

\paragraph{Shor's algorithm} $O(\log^2 n)$ gates to factor.

\paragraph{Diffie-Hellman Integrated Encryption Scheme (DHIES)} elgamal encryption with a MAC.

\section{Signatures}

\paragraph{RSA malleability} given $c = m^e \mod n$ for an unknown $m$, for any $x \in \mathbb{Z}_n^*$, we can construct $c' = (x^e \cdot c) \mod n = (xm)^e \mod n$. Selective forgery under a chosen message attack.

\paragraph{Full-domain hash RSA (FDH-RSA)} compute a signature on the hashed message. Hash function needs to be collision/preimage/second preimage resistant.

\paragraph{Digital Signature Algorithm (DSA)}

\begin{itemize}
  \item Setup:
  \begin{itemize}
    \item A prime $p$, a prime $q$ dividing $p-1$, an element $g \in \mathbb{Z}_p^*$ of order $q$, a hash function $H: \{0,1\}^* \to \mathbb{Z}_q$
  \end{itemize}
  
  \item Key generation:
  \begin{itemize}
    \item Choose $\alpha \in_R \mathbb{Z}_q^*$ at random. Return $(k_{\text{pubkey}}, k_{\text{privkey}}) = (g^\alpha \mod p, \alpha)$
  \end{itemize}
  
  \item Signing: To sign a message $m \in \{0,1\}^*$,
  \begin{itemize}
    \item Choose $k \in_R \mathbb{Z}_q^*$ at random
    \item Calculate $r = (g^k \mod p) \mod q$ and $s = \frac{H(m) + \alpha r}{k} \mod q$
    \item Repeat if $k$, $r$, or $s$ are zero. Otherwise, return signature $\sigma = (r,s)$
  \end{itemize}
  
  \item Verification: Given $k_{\text{pubkey}} = g^\alpha$, $m$, and $\sigma = (r,s)$,
  \begin{itemize}
    \item Check $0 < r < q$ and $0 < s < q$ and $(g^{\frac{H(m)}{s}} g^{\frac{\alpha r}{s}} \mod p) \mod q = r$
  \end{itemize}
\end{itemize}

\paragraph{Elliptic Curve Digital Signature Algorithm (ECDSA)}

\begin{itemize}
  \item Setup:
  \begin{itemize}
    \item A prime $p$, a prime $q$, an elliptic curve $E$ over $\mathbb{Z}_p$ of cardinality $|E| = q$, a generator $P \in E$ of order $q$, and a hash function $H: \{0,1\}^* \to \mathbb{Z}_q$.
  \end{itemize}
  
  \item Key generation:
  \begin{itemize}
    \item Choose $\alpha \in_R \mathbb{Z}_q^*$ at random.
    \item $(k_{\text{pubkey}}, k_{\text{privkey}}) = (\alpha P, \alpha)$
  \end{itemize}
  
  \item Signing: To sign a message $m \in \{0,1\}^*$,
  \begin{itemize}
    \item Choose $k \in_R \mathbb{Z}_q^*$ at random
    \item Calculate $r = x_kP \mod q$, and $s = \frac{H(m) + \alpha r}{k} \mod q$
    \item Repeat if $k$, $r$, or $s$ are zero.
    \item The signature is $\sigma = (r, s)$.
  \end{itemize}
  
  \item Verification: Given $\alpha P$, $m$, and $(r, s)$,
  \begin{itemize}
    \item Check $0 < r < q$ and $0 < s < q$
    \item Check that the $x$-coordinate of $\frac{H(m)}{s} P + \frac{r}{s} (\alpha P)$ is congruent to $r$ modulo $q$.
  \end{itemize}
\end{itemize}

\section{Key Management}

\paragraph{Certificates} subject identity, subject public key, issuer signature, expiration date etc.

\paragraph{TLS handshake} v1.3 generates a new ephemeral ECDSA key pair for each connection, signed by the server's certificate. Client always verifies server's key w its certificate. Optional client-to-server authentication.

\paragraph{Bitcoin} Signs transaction (including prev transaction id) using ECDSA key pair. proof of work by finding a hash value of sha-256 with 47 leading zeros.

\end{document}
